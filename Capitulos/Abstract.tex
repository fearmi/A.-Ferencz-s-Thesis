\section*{Abstract}
As the world’s population grows, farmers need to produce more and more food. Arable acreage cannot keep up, and the looming food insecurity threat could quickly devolve into regional or even global instability. To adapt, large farms are increasingly exploiting precision farming to increase yields, reduce waste, and mitigate the economic and security risks that inevitably accompany agricultural uncertainty.

The  Photogrammetry laboratory at TEC is committed to the technification of  Costa Rica's agriculture industry. It is possible to generate knowledge that improves time management, reduce water and chemical use, and produce healthier crops and higher yields, by acquiring precise data of the state of different agriculture fields with the help of stationary or robot-mounted sensors and camera-equipped drone.


This thesis aims to generate a methodology that provides the knowledge and tools to execute an autonomous photogrammetric mission with an open source fixed-wing UAS.  This methodology consists of four different steps. These steps guide the user from the planning and design of the flight stages to the integration of software and hardware up data processing.

Our experiment shows that with the proposed methodology, it is possible to obtain the data autonomously, therefore enabling the end user to generate different photogrammetric products. We suggest the use of Paparazzi open source platform to manage the UAS. Also, the use of  Apogee autopilot enable us to fully integrate the autopilot with the JeVois camera, giving the UAS the capability of collecting images and creating a GPS and attitude of the UAV data log, with the possibility to do image processing onboard.
Finally, we discuss how to improve the current UAV by suggesting several upgrades.

\textbf{\textit{Keywords}}: Drones, Fixed-wing, open source, mapping, photogrammetry, Paparazzi, mission planning.