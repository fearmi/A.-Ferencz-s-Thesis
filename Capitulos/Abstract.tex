\section{Abstract}
The lack of capability of performing mission over large study areas limits the capacity of the Photogrammetry laboratory at TEC. One of the main tasks of the laboratory is to obtain precise data of the current state of different agriculture fields to estimate and reduce the use of pesticides and add-ons that harm the environment. This is accomplished by performing a careful design mission, executed on an autonomous and controlled manner.

Conventional farming makes use of pesticides to protect plants and fertilizers to enhance their growth and fertility. They include herbicides to kill weeds, fungicides to get rid of diseases and insecticides to kill bugs. Those chemicals are unfortunately not only getting rid of the unwanted but can also cause harm to the health and the environment.

The aim of this study is to generate a methodology that provides the knowledge and tools to execute an autonomous photogrammetric mission with an open source fixed-wing UAS of a large area of study. This is accomplished in four different stages: planning, execution, programming and validating. The methodology is constructed by developing algorithms to design missions to survey area taking into consideration different parameters like GSD, front and end overlap and more. Once the mission is designed, the integration and characterization of the fixed-wing UAS take place. After completing the design phase its time to execute the mission to obtain data,  process it, and analyze.