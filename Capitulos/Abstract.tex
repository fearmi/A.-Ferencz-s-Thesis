\section{Abstract}
Conventional farming makes use of pesticides to protect plants and fertilizers to enhance their growth and fertility. They include herbicides to kill weeds, fungicides to get rid of diseases and insecticides to kill bugs. Those chemicals are not only getting rid of the unwanted but unfortunately, can also cause harm to the population health and the environment.

The  Photogrammetry laboratory at TEC is committed to the reduction of the impact of the agriculture industries on the Costa Rican environment. It is possible to generate knowledge that helps reduce pesticides and add-ons that harm the environment, By acquiring precise data of the state of different agriculture fields. This knowledge is obtained by performing missions with a  very specific design, executed on an autonomous and controlled manner.


This thesis aims to generate a methodology that provides the knowledge and tools to execute an autonomous photogrammetric mission with an open source fixed-wing UAS.  This methodology consists of three different steps. These steps guide the user from the planning and design of the flight to the integration of software and hardware and finish with the processing of the data.

Our experiment shows that with the proposed methodology it is possible to obtain the data autonomously. Therefore we enable the end user to generate different photogrammetric products. We propose the use  Paparazzi open source platform to manage the UAS. The use of Apogee autopilot enables us to fully integrate the autopilot with the JeVois camera, giving the UAS the capability of collecting both images and create a data log with the GPS and attitude of the UAV, even leaving the possibility to do image processing onboard.
Finally, we discuss how to improve the current UAV by suggesting different upgrades.