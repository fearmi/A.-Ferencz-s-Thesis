\section*{Resumen}
A medida que la población mundial aumenta, los agricultores necesitan producir cada vez más alimentos. La superficie cultivable no puede seguir el ritmo, y la amenaza a la seguridad alimentaria que se avecina podría llevar rápidamente a la inestabilidad regional o incluso mundial. Para adaptarse, las granjas grandes están explotando cada vez más la agricultura de precisión para aumentar los rendimientos, reducir el desperdicio y mitigar los riesgos económicos y de seguridad que inevitablemente acompañan a la incertidumbre agrícola.

El laboratorio de fotogrametría de TEC está comprometido con la tecnificación de la industria agrícola de Costa Rica. Es posible generar conocimiento que mejore la gestión del tiempo, reduzca el uso de agua y productos químicos, y produzca cultivos más sanos y rendimientos más altos, mediante la adquisición de datos precisos del estado de los diferentes campos agrícolas con la ayuda de sensores estacionarios o montados en robots y \textit{Drones} equipados con camaras.

Esta tesis tiene como objetivo generar una metodología que proporcione los conocimientos y las herramientas para ejecutar una misión fotogramétrica autónoma con un UAS de ala fija uso libre con. Esta metodología consta de tres pasos diferentes. Estos pasos guían al usuario desde la planificación y el diseño del vuelo hasta la integración de software y hardware y finalizan con el procesamiento de los datos.

Nuestro experimento muestra que con la metodología propuesta es posible obtener los datos de forma autónoma. Por lo tanto, permitimos al usuario final generar diferentes productos fotogramétricos. Proponemos el uso de la plataforma de código abierto Paparazzi para gestionar los UAS. El uso del piloto automático Apogee nos permite integrar completamente el piloto automático con la cámara JeVois, brindando al UAS la capacidad de recopilar ambas imágenes y crear un registro de datos con el GPS y la actitud del UAV, incluso dejando la posibilidad de realizar el procesamiento de imágenes a bordo.
Finalmente, discutimos cómo mejorar el UAV actual sugiriendo diferentes actualizaciones.

\textbf{\textit{Palabras claves}}: Drones, Fixed-wing,software de uso libre, mapeo, photogrammetría, Paparazzi, planificación de mision.