\section{Conclusions and Recommendations}
The workflow proposed in this thesis allowed for the creation of photogrammetric products using an open source fixed-wing UAS, by following a set of steps that were executed methodologically — allowing the end user to obtain results in a controlled and autonomous fashion.

Using the open source Paparazzi ground control station platform enabled the user to design a flight path, taking into consideration predefined parameters and requirements. Using an Apogee autopilot in conjunction with Paparazzi it was possible to execute the mission in a controlled and autonomous fashion.

The lack of quality of the camera and its picture hade a critical impact on the capacity to execute a correct photo-alignment. This situation, combined with the lack of the extrinsic and intrinsic parameters of the camera, represents a significant error source impacting the capabilities of the image stitching software to predict the correct altitude.

 After carrying out the methodology, using an Opterra 2 UAV and JeVois camera, it was possible to obtain an orthomosaic and digital elevation model of 1.3 cm/pixel.
 

It is recommended that a camera design for photogrammetric purposes be used. Upgrading to a global shutter camera with geotagging capabilities will improve dramatically the quality of the results obtained with the methodology.

The LiPo batteries are sensitive to the outside temperature, as the temperature falls the storage capabilities are reduced. Also, this kind of equipment is designed to have a high current output given the user the capabilities of driving several motors a the same time but there is a trade-off in weight. Fixed-wing UAV only have 1 or 2 engines. The use of LiOn batteries can increase the autonomy of the system by reducing the takeoff load