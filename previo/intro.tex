\section{Introduction}
As the global population recently crossed 7-billion mark, the tendency towards an increase in the population becomes very important and tangible problem. Due to this fact, humanity needs to provide a noticeable increase in agricultural products output. Lack of farming lands, high labor costs, changes in climate, crop losses due to pests bring additional challenges that limit access to the good-quality food for people in developing countries\cite{7551564}

As a country that relies heavily on both agriculture and ecotourism, Costa Rica's utilization of pesticides (agrochemicals) requires striking a delicate balance between maximizing crop production and minimizing the negative effects of pesticide pollution. Costa Rica's heavy reliance on pesticides began in the 1940s when foreign corporations introduced pesticide use to Costa Rican farmers and pesticide reliance quickly began to grow. In 2000, The World Resources Institute issued a report listing Costa Rica as the highest consumer of pesticides per hectare of agricultural land in the world, using on average an alarming 18.2 kilograms of pesticides per hectare of cropland. While many farmers around the country attest that pesticides are absolutely necessary for crop production, the effects of such a heavy-handed approach to agriculture on both the environment as well as on the residents of the country cannot be overlooked. \cite{UNA}.

Assessment of the health of a crop, as well as early detection of crop infestations, is critical in ensuring good agricultural productivity. Stress associated with, for example, moisture deficiencies, insects, fungal and weed infestations, must be detected early enough to provide an opportunity for the farmer to mitigate. If the process is done in a manual matter it will take at list two weeks, living a very small window of opportunity to make the necessary adjustments. On the other hand, if this process utilizes remote sensing imagery with a UAS, the results can be delivered to the farmer quickly, usually within 2 days.\cite{crops}

Also, crops do not generally grow evenly across the field and consequently crop yield can vary greatly from one spot in the field to another. These growth differences may be a result of soil nutrient deficiencies or other forms of stress. Remote sensing with UAS, allows the farmer to identify areas within a field which are experiencing difficulties, so that he can apply, for instance, the correct type and amount of fertilizer, pesticide or herbicide. Using this approach, the farmer not only improves the productivity from his land but also reduces his farm input costs and minimizes environmental impacts.\cite{crops}

%There are many people involved in the trading, pricing, and selling of crops that never actually set foot in a field. They need information regarding crop health worldwide to set prices and to negotiate trade agreements. Many of these people rely on products such as a crop assessment index to compare growth rates and productivity between years and to see how well each country's agricultural industry is producing.\cite{crops}

Agricultural UAS technology has been improving in the last few years, and the benefits of UAS in agriculture are becoming more apparent to farmers. Drone applications in agriculture range from mapping and surveying to crop dusting and spraying. On the surface, agricultural UAS are no different from other types of drones. The application of the UAS simply changes to fit the needs of the farmer. There are, however, several UAS specifically made for agricultural use, for example, Precision Agriculture, mapping/surveying, crop dusting/spraying. 

There are different options in the market for agriculture UAS, ranging from \$5000 to \$10000, representing a big challenge to small producers in developing countries. By using open source hardware and software, it is possible to make this kind of technology widely available to the agriculture industries. Reducing the impact of there activities on the environment.

\subsection{Outline and main contributions}
In chapter 3 we briefly introduce the basic terminology used in the report and the concept of UAS, its components, and basic functionality to have a better understanding of the expected behavior .

Chapter 4 will briefly describe the principles and workflow of computer vision. Also in the chapter is an introduction to JeVois camera and capabilities

Chapter 5 define terms relative to mission planning, then creates a relationship between the term using mathematical equation.  The chapter finish with an explanation of the surveying routine executed in the flights.

