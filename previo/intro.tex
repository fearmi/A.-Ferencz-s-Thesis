\section{Introduction}
Photogrammetry is a technique with utility in very diverse applications such as agriculture, forestry, topography, an inspection of structures, among many others. This technique allows for the making of geo-referenced three-dimensional maps from a series of photographs of the area under study.\cite{Articulo}. To carry out a photogrammetric survey, it is necessary to take photographs with a certain overlap to completely cover the terrain to be analyzed. It is also essential to gather data of the time, position and orientation of each of the photographs for further processing, which will lead to photogrammetric products. The implementation of this technique with unmanned aerial systems (UAS) represents a simple, low cost, high resolution solution compared to other alternatives as is the use of manned aircraft and satellites.\cite{FotogrametryThesis}

Currently  the photogrammetry laboratory (UASTEC) of the School of Electronic Engineering of the Technological Institute of Costa Rica (TEC) is investigated the use of photogrammetric techniques with UAS systems for various applications. The Lab has several multirotor platforms that have great maneuverability and agility. However, due to the high power consumption associated with the multiples engines, autonomy is severely punished.\cite{Leonardo}. In the past years, the lab has been exploring the use of fixed-wing UAV (Unmanned Aerial Vehicle) for photogrammetric purposes. The graduation project of engineer  Edgar Guiterrez accomplished the first step in this process, by developing a Fixed-wing UAV for photogrammetric use.\cite{Edgar}\cite{conference}

Unlike multi-rotor platforms, fixed-wing systems are characterized by having a more straightforward electronic system, so its energy consumption is lower. Its physical structure ensures a more efficient aerodynamic performance, which allows them to achieve high flight speeds and therefore cover more area. Together, these two properties profile fixed-wing systems as one alternative for large-scale analysis.

With the help of Ph.D. Héctor Garcia de Marina Peinado, and  his team at the SDU UAS center, this project aims to develop a methodology that will allow the end user to design and follow a flight path with an open source fixed-wing UAV in order  to gather the necessary data to create photogrammetric products. The methodology must have the capability of computing the parameters required to ensure an optimized design in order to meet the photogrammetric product requirements.

 \textit{\textbf{Document structure}}
In chapter 2 we briefly introduce the basic terminology used in the report and the concept of UAS, its components, and basic functionality to have a better understanding of the expected behavior also introduce term relative to photogrametry.

Chapter 3 define terms relative to mission planning, then creates a relationship between the term using a mathematical equation — the chapter finish with an explanation of the surveying routine executed in the flights.

Chapter 4 starts with a general description of the methodology, stating the inputs, outputs of the modules that compose the workflow. Next, in this chapter, there is an insight into each module explaining each sub-block of the module.

Chapter 5 is where the validation of the proposed methodology takes place. 


