\section{Introduction}

In the recent year, the use of unmanned aerial systems (UAS) and remote sensing applications have been increasingly spreading out. This success is mainly due to the flexibility and the ability to gather high-resolution geometric data, also because of the impressive diffusion of Unmanned Aerial Systems (UAS) in the fields of aerial photogrammetry. The use of such versatile and flexible platforms has led to a new and important revitalization of techniques of acquisition and processing of data from airborne platforms. \cite{Planning_airborne} 

The UAS platforms used for photogrammetry applications are known as mapping drones, usually with the ability to perform autonomous flights. The acquisition of data through manual flight is nearly impossible due to the difficulties faced when controlling and synchronizing the flight path with remotely acquired data. An automated flying system scans the area under test (AUT) on a control and systematic matter, executing a preprogrammed flight plan.

Mapping drones can be categorized into two categories: Multirotor and fixed wing. The first one resembles a helicopter with multiple rotors, meanwhile, the fixed wings operate under the aerodynamics principal of Bernoulli’s law. Multirotor systems are usually composed of 4, 6 or 8 rotors, are highly maneuverable and capable of hovering. Due to the ability to overcome obstacles, it is possible to perform low altitude flights, resulting in better ground sampling resolution (GSR) products, however, these platforms are slower and have limited autonomy due to the multiple rotors working at the same time. Fixed wing platforms usually have one or two motors which provide them with a greater autonomy, flying at greater speed, but sacrificing GSR.\cite{pulicacion}.

When surveying a very large AUT for a Geographic Information system (GIS) analysis, there are several methods: manual surveying, surveying using a multirotor UAS and surveying using a fixed-wing UAS platform.

The use of manual techniques can be very precise, but it would be nearly impossible to perform on a large scale. When surveying large areas, the use of multirotor UAS will be very limited due to the reduced autonomy characteristic of this kind of platforms.

The Photogrammetry Lab at TEC has a joined project with CORBANA (Corporación Bananera Nacional), which strive the progress of the banana sector in Costa Rica, and the community’s that surround the sector, generating work options and boosting the development and production of the fruit, reducing the environmental impact of the plantations. One of the measures to accomplish this is with photogrammetric studies that allow the reduction of agrochemicals and pesticides used. The use of a fixed-wing platform is key to the success of a study in places with this type of conditions\cite{CORBANA}.

This report presents the design of a methodology to generate and tracking of trajectories for photogrammetric surveys with a fixed-wing platform. The problem consists of the lack of a methodology to execute a large-scale photogrammetric study with multirotor UAS platforms like the one used in the photogrammetric lab. 



The main objective of this project is to design a methodology for execution of automated photogrammetric mission with fixed-wing UAS platforms. This would be achieved by designing flight routes based on different characteristics that will allow generating photogrammetric products, executing the methodology design and then validating the methodology with repetition of flight in different conditions

